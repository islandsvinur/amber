\section{\label{sct:messages}Messages}

There are a few message types in the system. Two of which must be defined
system-wide because they are used in the communication between modules.

\subsection{\label{sct:messages:story}Message type `Story'}

The Story message is only posted to the whiteboard `WB.Stories' and only by
Crawler modules.

The message content is a \ac{YAML}\footnote{\url{http://www.yaml.org/}} document
which represents the storyData field inside the Java counterpart of the
message. It contains at least the properties `URI' (to identify the story,
GUIDs are RSS specific and cannot be used), `Author', `Title', `Story-Content'.

It may also contain `Publication-Date' (which is the date of publication in
Internet~Message~Format\cite[Section~3.3]{RFC2822}), `Kind' and other fields.

An example of a YAML document containing Story data:

\begin{verbatim}
---
URI: http://ijsland.luijten.org/2006/09/12/skyr-wasdanou/
Author: Christian Luijten
Title: Skyr... Wasdanou?
Publication-Date: Tue, 12 Sep 2006 21:03:54 +0000
Kind: weblog-posting
Story-Content: >
  Een van die dingen die bij een onbekende cultuur horen zijn de
  eetgewoonten. Elk land heeft zo z’n producten die je nergens
  anders kan krijgen. IJslands nationale zuivelproduct heet Skyr,
  elke oma kan het maken, al is het nogal een hoop werk. Daarom is
  het lange tijd (lees: gedurende de jachtige periode na de tweede
  wereldoorlog toen de Amerikanen hier de boel kwamen ophaasten)
  in ongebruik geraakt, maar op een gegeven moment kwam de vraag
  toch weer terug en zijn een aantal zuivelproducenten het
  industrieel gaan produceren.
\end{verbatim}

\subsection{Message type `Analysis'}

Analysis typed messages are posted on the `WB.Analyses' whiteboard only by
Sieve modules. They contain information about stories present on the
`WB.Stories' whiteboard.

The content of these messages is also YAML format. Story messages are coupled
with Analysis messages through their `URI' fields, so this must be present.

An example of a message issued by an analysis module checking for the topic
`Zuivel' (which means dairy products in Dutch):

\begin{verbatim}
---
URI: http://ijsland.luijten.org/2006/09/12/skyr-wasdanou/
Topic: Zuivel
Relation-Strength: 1.0
Author-Strength: 0.1
\end{verbatim}

Its `Relation-Strength' suggests high relevance of the content with the topic.
However, the `Author-Strength' suggests that the author isn't an authority in
the field.

Every analysis module sends a message to the whiteboard if it thinks it is
relevant. It is thus possible that the same URI will get multiple analysis
results or nothing at all, the visualizer module must cope with this and merge
the available information.

