\chapter{\label{cpt:evaluation}Evaluation}

\section{Educational evaluation}

\subsection{Learning points}

The approach used in the development of this project is called
Constructionists Design Methodology, which is a form of incremental
design. For this project, this was a good choice, because many things were
undecided upon in the beginning and only became clear after some first
experiments.

One example would be the visualization of ``replies to a story'' which was
initially thought of as adding a particle on a tail tied to the original
story. The longer the tail, the more important a story would appear.

However, there is no connection between a reply and a story other than a
humanly constructed one (there is no field in RSS which specifies it). It
was thus impossible to create this link without going far beyond the scope
of this project.

Instead, it was chosen to give all replies their own orbiting particles
and if a story is popular, many particles appear and in this way visualize
activity in a topic.

\subsection{Points for improvement}

It is very hard to create a realistic planning and to estimate one's
capabilities right. Some details had to be simplified on the way to make
the project fit within the available time, while other aspects which
seemed complex were in fact solved very quickly.

\section{Technical evaluation}

\subsection{Unimplemented parts}

It is not yet possible to ``travel through the past'' as was an initial
idea. This feature would enable a user to set a certain time in the past
and see the situation at that point. A very simple approach would be to
create screenshots of the display for every day and then display the
desired one.

\subsection{Points for expansion}

The nice thing about using a message and whiteboarding system like
Psyclone is that every component can be replaced fairly easily.

Currently, there is only one analysis module which can not make very
intelligent decisions. The fact that this is a project within the AI
department, makes it likely that someone will implement a smarter analysis
module which can run in place or alongside the current one.

It is only possible to get stories from RSS feeds, but another module
could be created which gets information out of newsgroups. A lot of
AI discussion is going on on Usenet, so this might also be an interesting
expansion of the system.

If the applet is to be used in a forum, it can be that a module with
direct (read-only) database access is created.

Then, finally, the visualization itself can be replaced by another one and
one of the ideas is to make it look more like the Sun's corona, expanding
it in those directions where much discussion is going on, shrinking it
where discussion is quiet.
