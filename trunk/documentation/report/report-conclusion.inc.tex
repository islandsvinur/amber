\chapter{\label{cpt:conclusion}Conclusion}

During this internship, I created a simple framework for Internet discussion
analysis and visualization. Along with the framework, I provide three modules
that illustrate the possibilities within the framework.

Both modules and framework are extensible or replacable in various ways.
The current visualization has a strong emphasis on stories as an entity.
To get a visualization of all activity on the internet, a sample should be
taken and the orbiting particles will no longer be entities on their own.
They rather turn into classes of entities, which probably calls more for a
cloud-like visualization.

One of the ideas is to change Ambient Earth into Ambient Sun, where the
Sun's corona (the brightly shining part which is visible during a total
solar eclipse) is taking over the role of the particles orbiting Earth.

\section{Unimplemented parts}

It is not yet possible to ``travel through the past'' as was an initial
idea. This feature would enable a user to set a certain time in the past
and see the situation at that point. A very simple approach would be to
create screenshots of the display for every day and then display the
desired one.

\section{Points for expansion}

The nice thing about using a message and whiteboarding system like
Psyclone is that every component can be replaced fairly easily.

Currently, there is only one analysis module which can not make very
intelligent decisions. The fact that this is a project within the AI
department, makes it likely that someone will implement a smarter analysis
module which can run in place or alongside the current one.

It is only possible to get stories from RSS feeds, but another module
could be created which gets information out of newsgroups. A lot of
AI discussion is going on on Usenet, so this might also be an interesting
expansion of the system.

If the applet is to be used in a forum, a module with direct (read-only)
database access could be created.

The applet could also be used in software development, displaying for
instance build status, bugreports assigned to programmers, server load
etc.

Then, finally, the visualization itself can be replaced by another one and
one of the ideas is to make it look more like the Sun's corona, expanding
it in those directions where much discussion is going on, shrinking it
where discussion is quiet. 
