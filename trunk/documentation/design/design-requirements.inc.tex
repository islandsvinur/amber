\chapter{\label{cpt:requirements}Requirements}

There are various kinds of requirements to be identified. A distinction can be
made between functioanl and extra-functional (or non-functional) requirements.

\section{Extra-functional requirements}

\begin{enumerate}
  \item The system must make use of the Psyclone framework for communication.
  \item The system will be implemented in the Java programming language.
  \item The display module with the Java Applet must be able to run on any
        machine with the Java plugins properly installed (i.e. not only on the
        machine running the rest of the system).
  \item It must be possible to add modules with similar functionality to
        operate in parallel with modules already there. For instance when the
        Java Applet is running, it should also be possible to have the full
        screen module running at the same time. 
\end{enumerate}

\section{Functional requirements}

These requirements describe which \emph{inputs}, \emph{outputs}, \emph{storage}
and \emph{computations} exist in the system and how they are \emph{timed and
synchronized}.

\subsection{Inputs}

\begin{enumerate}
  \item The system must be configurable to specify which RSS feeds will be
        monitored.
  \item The configuration of the system does not have to be done during run-time.
  \item Changing the configuration of a module may require a restart of the
        module.
  \item The system will use the configuration to get information from the
        internet from the specified RSS feeds.
  \item The display may have a set of controls to navigate through the history
        of a feed.
  \item Parts of the system must accept triggers from Psyclone whiteboards.
\end{enumerate}

\subsection{Outputs}

\begin{enumerate}
  \item There will be an output module which is to be used within a website, i.e. a Java Applet.
  \item There will be an output module which can display more information than
        the Applet which runs in full screen.
\end{enumerate}

\subsection{Storage}

\begin{enumerate}
  \item The system does not store anything.
\end{enumerate}

\subsection{Computations}

\begin{enumerate}
  \item The system must decide of a delivered story what its subject(s) is/are.
  \item The system may put weights on the subjects instead of a boolean value.
\end{enumerate}

\subsection{Timing and synchronization}

\begin{enumerate}
  \item Synchronization between modules is handled by Psyclone, so no
        requirements need to be added to the system itself.
\end{enumerate}

