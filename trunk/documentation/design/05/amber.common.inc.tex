\section{Objects in the amber.common package}

\classname{AirBrush}

\begin{classmetadata}
  \function{Ease communication with Psyclone through JavaOpenAIR.}
\end{classmetadata}

\begin{interface}
  \init{AirBrush}{String moduleName, String hostName, int port}
    {Initializes a connection with Psyclone on \emph{hostName:port} for module
      \emph{moduleName}.}
  \method{\void}{setCallbackObject}{AirBrush\-Callable}
    {Sets the callback object to be used for calls from AirBrush.}
\end{interface}



\interfacename{AirBrushCallable}

\begin{interface}
  \method{\void}{airBrushCallBack}{com.cmlabs.air.Message msg}
    {Used as callback function for the AirBrush.}
\end{interface}



\classname{Story}

\begin{classmetadata}
  \function{Storage of Story data.}
  \data{Story has a Map$\langle$String, Object$\rangle$ field where it stores
    all data. It also holds a value with the number of analyses left until it
    is considered enough to be visualized. Instead of thrown back on the
    whiteboard with raw stories, it should then go the the processed stories.}
\end{classmetadata}

\begin{interface}
  \method{\void}{setAuthor}{String author}{Sets the author of the story.}
  \method{String}{getAuthor}{}{Gets the author of the story.}
  \method{\void}{setCreationTime}{Date time}{Set the creation time of the story}
  \method{Date}{getCreationTime}{}{Get the creation time of the story.}
  \method{\void}{setContent}{String text}{Set the content of the story.}
  \method{String}{getContent}{}{Get the content of the story.}
  \method{\void}{setValue}{String k, Object v}
    {Store an arbitrary object v under keyword k.}
  \method{Object}{getValue}{String k}
    {Gets the object stored under keyword k.}
  \method{Boolean}{lockForAnalysis}{}
    {Request a lock on the story for analysis. Returns true when given, assumes
      niceness of the analysis modules.}
  \method{\void}{unlock}{}
    {Removes the lock. Again, assumes niceness of the analysis modules.}
  \method{Boolean}{isAnalysisDone}{}
    {Returns true when the story finds it is analysed enough and doesn't need
      another run.}
  \method{void}{fromYAML}{String yaml}
    {Parses a \ac{YAML} string (e.g. coming from Psyclone) to the contents of the
      object (overwrites values currently stored!). Note, together with toYAML,
      this is the identity function: story.fromYAML(story.toYAML()) = story.}
  \method{String}{toYAML}{}
    {Converts the contents of the object to a \ac{YAML} string to be sent via
      Psyclone.}
\end{interface}

