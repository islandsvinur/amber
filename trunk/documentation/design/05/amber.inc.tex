\section{Objects in the amber package}

The classes in the main package are launchers for the three different modules
and abstract classes for the main objects of these modules. Basically it works
like this; if the Crawler class is started it creates a CrawlerObject and
starts it.


\classname{Crawler}

\begin{classmetadata}
  \function{The Crawler object makes launching a crawler easy and uniform for
      all crawler types.}
\end{classmetadata}

\begin{interface}
  \init{Crawler}{}
    {Creates an instance of Crawler. Also creates a CrawlerObject and the
      connection to Psyclone.}
  \method{\static\ \void}{main}{String\[\] args}
    {Method executed upon launch of the Crawler.}
  \method{\void}{start}
    {Starts the Crawler.}
  \method{\void}{stop}
    {Stops the Crawler.}
\end{interface}



\abstractclassname{CrawlerObject}

\begin{classmetadata}
  \function{The CrawlerObject class provides a uniform interface for crawler
      applications.}
\end{classmetadata}

\begin{interface}
\end{interface}



\classname{ShowOff}

\begin{classmetadata}
\end{classmetadata}

\begin{interface}
\end{interface}



\abstractclassname{ShowOffObject}

\begin{classmetadata}
\end{classmetadata}

\begin{interface}
  \method{\void}{setStoryQueue}{\mbox{Queue$\langle$Story$\rangle$} q}
    {Sets the queue of incoming stories. ShowOff will handle communication with
      the Psyclone whiteboard and puts stories in the queue.}
\end{interface}



\classname{Sieve}

\begin{classmetadata}
\end{classmetadata}

\begin{interface}
\end{interface}



\abstractclassname{SieveObject}

\begin{classmetadata}
\end{classmetadata}

\begin{interface}
\end{interface}

