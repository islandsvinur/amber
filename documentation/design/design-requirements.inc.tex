\chapter{\label{cpt:requirements}Requirements}

There are various kinds of requirements to be identified. A distinction can be
made between functional and extra-functional (or non-functional) requirements.

\section{Extra-functional requirements}

\begin{enumerate}
  \item The system must make use of the Psyclone framework for communication.
  \item The system will be implemented in the Java programming language.
  \item The display module with the Java Applet must be able to run on any
        machine with the Java plugins properly installed (i.e. not only on the
        machine running the rest of the system).
  \item It must be possible to add modules with similar functionality to
        operate in parallel with modules already there. For example when the
        Java Applet is running, it should also be possible to have the full
        screen module running at the same time. 
\end{enumerate}

\section{Functional requirements}

These requirements describe which \emph{inputs}, \emph{outputs}, \emph{storage}
and \emph{computations} exist in the system and how they are \emph{timed and
synchronized}.

\subsection{Inputs}

\begin{enumerate}
  \item The system must be configurable to specify which sources will be
        monitored.
  \item The system will use the configuration to get information from the
        internet from the specified sources.
  \item Configuration of the system goes via Psyclone using module parameters.
  \item The display may have a set of controls to navigate through the history
        of a feed.
  \item Parts of the system must accept triggers from Psyclone whiteboards.
  \item Sources must be \ac{RSS} feeds. The system should however be prepared
        to support other source types as well (i.e. it should be easily
        expandable).
  \item The Applet display is non-interactive (no input).
\end{enumerate}

\subsection{Outputs}

\begin{enumerate}
  \item There is an output module which is to be used within a website,
        i.e. a Java Applet.
  \item There is an output module which runs standalone and in full screen
        and displays more information than the Applet can.
\end{enumerate}

\subsection{Storage}

\begin{enumerate}
  \item The system on itself does not store anything.
\end{enumerate}

\subsection{Computations}

\begin{enumerate}
  \item The system must decide of a delivered story what its subject(s) is/are.
  \item The system may put weights on the subjects instead of a boolean value.
\end{enumerate}

\subsection{Timing and synchronization}

\begin{enumerate}
  \item Synchronization between modules is handled by Psyclone, so no
        requirements need to be added to the system itself.
\end{enumerate}

\subsection{Visualization}

\subsubsection{Common}

The following requirements are common for both the applet and the standalone
viewer.

\begin{enumerate}
  \item A story is represented as a dot.
  \item Dots fly around freely in the display.
  \item Dots are attracted by points in the space which represent concepts the
        stories can be about.
  \item Dots are ``spewed out'' from the middle, from where they slowly move
        outward, in the direction of their ``attraction'' points.
\end{enumerate}

\subsubsection{Applet}

\begin{enumerate}
  \item
\end{enumerate}

\subsubsection{Standalone viewer}

\begin{enumerate}
  \item
\end{enumerate}
