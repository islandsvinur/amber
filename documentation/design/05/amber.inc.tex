\section{Objects in the amber package}

The classes in the main package are launchers for the three different modules
and abstract classes for the main objects of these modules. Basically it works
like this; if the Crawler class is started it creates a CrawlerObject and
starts it.

\abstractclassname{AmberObject}
\begin{classmetadata}
  \implements{Runnable}
  \function{Provide a consistent interface and functionality to all three Amber
    modules.}
  \processing{AmberObject has functionality for running a module in a thread,
    which is most likely to be necessary. Further it has methods for graceful
    start and stop.}
\end{classmetadata}
\begin{interface}
  \method{\void}{start}{}
    {Starts the module.}
  \method{\void}{stop}{}
    {Stops the module.}
  \method{\void}{run}{}
    {Needed for implementation of Runnable interface.}
\end{interface}



\classname{Crawler}

\begin{classmetadata}
\end{classmetadata}

\begin{interface}
\end{interface}



\abstractclassname{CrawlerObject}

\begin{classmetadata}
  \extends{AmberObject}
\end{classmetadata}

\begin{interface}
\end{interface}



\classname{ShowOff}

\begin{classmetadata}
\end{classmetadata}

\begin{interface}
\end{interface}



\abstractclassname{ShowOffObject}

\begin{classmetadata}
  \extends{AmberObject}
\end{classmetadata}

\begin{interface}
  \method{\void}{setStoryQueue}{Queue$\langle$Story$\rangle$ q}
    {Sets the queue of incoming stories. ShowOff will handle communication with
      the Psyclone whiteboard and puts stories in the queue.}
\end{interface}



\classname{Sieve}

\begin{classmetadata}
\end{classmetadata}

\begin{interface}
\end{interface}



\abstractclassname{SieveObject}

\begin{classmetadata}
  \extends{AmberObject}
\end{classmetadata}

\begin{interface}
\end{interface}

