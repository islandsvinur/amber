\section{\label{sct:modules}Modules}

A running Amber system will comprise of at least three modules running at the
same time; there is a Crawler module, a selector module called Sieve and a
display called ShowOff.

Every module has a specified interface through which communication with
Psyclone is handled.

\subsection{Crawler modules}

\subsubsection{RSS}

The RSS crawler module will be fairly straightforward. There are actually quite
a few good RSS parsers around for Java, the only thing the crawler should do is
getting the stories from the RSS feeds along with meta-information and store it
on the whiteboard.

\subsection{Sieve modules}

\subsubsection{KeywordSpotter}

The KeywordSpotter is configured to detect the presence of certain keywords in
a story which makes it fit in a certain category or subject. In early versions
the weights of the subjects will be equal. In later versions all weights of all
subjects of a story must for instance add up to 100\%. This gives the
visualizer more freedom to place the story.

\subsubsection{PhraseSpotter}

The PhraseSpotter will use a grammar engine to detect whether certain types of
sentences appear in a story to classify it.

\subsection{ShowOff modules}

\subsubsection{Full screen}

The full screen application will display a lot of information and is there to
be looked - not glanced - at. It should be possible to let it do its job
autonomously, just showing a pretty picture, or to be interactive.

\subsubsection{Ambient applet}

The ambient applet will display a very easy to understand image of the status
of the page it is on. I.e. if the page is a weblog, it should display subject
information on that weblog, if it is on the page of a thread of a forum, it
displays the flow of the discussion.

