\chapter{\label{cpt:requirements}Requirements}

There are various kinds of requirements to be identified. A distinction can be
made between functional and extra-functional (or non-functional) requirements.

\section{Extra-functional requirements}

\begin{enumerate}
  \item The system must make use of the Psyclone framework for communication.
  \item The system will be implemented in the Java programming language.
  \item The display module with the Java Applet must be able to run on any
        machine with a properly installed Java Virtual Machine (i.e. not only
        on the machine running the rest of the system).
  \item It must be possible to add modules with similar functionality to
        operate in parallel with modules already there. For example when the
        Java Applet is running, it should also be possible to have the full
        screen module running at the same time. 
\end{enumerate}

\section{Functional requirements}

These requirements describe which \emph{inputs}, \emph{outputs}, \emph{storage}
and \emph{computations} exist in the system and how they are \emph{timed and
synchronized}. Finally, since this is a very important part of the project,
there are two separate sections on \emph{story analysis} and
\emph{visualization} requirements.

\subsection{Inputs}

\begin{enumerate}
  \item The system must be configurable to specify which sources will be
        monitored.
  \item The system will use the configuration to get information from the
        internet from the specified sources.
  \item Configuration of the system goes via Psyclone using module parameters.
  \item The display may have a set of controls to navigate through the history
        of a feed.
  \item Parts of the system must accept triggers from Psyclone whiteboards.
  \item Sources must be \ac{RSS} feeds. The system should however be prepared
        to support other source types as well (i.e. it should be easily
        expandable).
  \item The Applet display is non-interactive (no input).
\end{enumerate}

\subsection{Outputs}

\begin{enumerate}
  \item There is an output module which is to be used within a website,
        i.e. a Java Applet.
  \item There is an output module which runs standalone and in full screen
        and displays more information than the Applet can.
\end{enumerate}

\subsection{Storage}

\begin{enumerate}
  \item The system on itself does not store anything.
\end{enumerate}

\subsection{Computations}

\begin{enumerate}
  \item The system must decide of a delivered story what its subject(s) is/are.
  \item The system may put weights on the subjects instead of a boolean value.
\end{enumerate}

\subsection{Timing and synchronization}

Synchronization between modules is handled by Psyclone, so no requirements need
to be added to the system itself.

\subsection{Story analysis}

\begin{enumerate}
  \item When stories come in, they are analysed by analysis modules.
  \item Every module adds some meta-information to the story depending on the
        module analysis.
\end{enumerate}

\subsection{Visualization}

The following requirements are common for both the applet and the standalone
viewer.

\begin{enumerate}
  \item A story is represented as a dot.
  \item In the center of the display is Earth (with picture?).
  \item Dots are launched into orbit around Earth.
  \item The orbits follow Kepler's laws of planetary motion.
  \item The launch velocity is dependent on how ``big'' the story is (like from
        an important author or if it has many references) at the time of
        writing, but is always smaller than the escape velocity.
  \item Stories get velocity boosts by getting replies/comments or
        references (``trackbacks'') in order to keep them around longer.
  \item Stories which don't get reactions will just orbit Earth for a while and
        eventually fall back down \mbox{(i.e. launch velocity $<$ escape
        velocity)}.
  \item Only the meta-information added to the stories by the Sieve modules is
        used to determine launch variables.
  \item There are some small, heavy bodies in geostationary orbit around Earth
        representing values of an enumeration of meta-information (for instance
        story subjects). They attract the stories depending on how much they
        match the story's subject. It is possible for a story to get into orbit
        with such a heavy body if it is really strongly connected to the
        subject.
\end{enumerate}

There will be two different views, a static and a dynamic one. Which one is
used depends on the application. To get an idea of the activity at a certain
moment in time, the static view is used. For a ``real-time'' view of internet
activity, the dynamic view can be used.

The term ``static'' doesn't mean the image is standing still, it will behave
exactly the same as the dynamic view. However, some physical laws don't apply
or are differently calibrated in order to give a constant image. In other
words, while in dynamic view stories can appear and disappear, in the static
view the stories are a given constant.

\subsubsection{Differences between Applet and Standalone viewer}

\begin{enumerate}
  \item The applet display will in practice be considerably smaller than the
        standalone viewer. Therefore, the applet is less detailed and some
        physical laws might need to be bend a bit.
\end{enumerate}

