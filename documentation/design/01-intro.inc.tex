\chapter{\label{cpt:design:introduction}Introduction}

This document describes the design of the \AmbE\ project. The ambitional goal
of this project is to give an ambient view on the activity on the whole
world-wide web. In practice, it shows the activity on for instance a forum or
larger weblog system.

The name of the system is \Amber{}.

The design of the project will follow the Constructionist Design Methodology
for Interactive Intelligences\cite{CDM}. First, a few usage scenarios are given
in Chapter~\ref{cpt:scenarios}. These scenarios result in the requirements
which are listed in Chapter~\ref{cpt:requirements}. Using the requirements, an
architecture is written up in Chapter~\ref{cpt:architecture}.

Please note that in this document some basic knowledge of the terminology of
Psyclone framework is needed, which is given in the next section. For more
information about Psyclone, refer to the full
documentation\cite{PsycloneManual}.

\section{Quick introduction to Psyclone}

CMLabs, the creator of Psyclone says this about their product:

\begin{quote}
  Psyclone™ is a powerful platform for building modular, distributed systems.
  It is the middleware of choice in systems where complexity management or
  interactivity is key.
\end{quote}

For this project, it is enough to know that there are ``modules'' and
``whiteboards''. Via a specification file the user can decide which types of
messages will be coming from which modules to which whiteboards and more
importantly, which types of messages on which whiteboard will trigger an event
in which module. There are many more possibilities with the system, but this is
all functionality \Amber\ is using.

There are two types of modules, internal and external. We are only interested
in external modules right now.

Below is an example of the specification of a module, in this case of the
Applet module. 

\begin{verbatim}
<module name="Module.ShowOff.Applet.Anonymous">
 <executable />
 <description>
  This module gets stories from the processed whiteboard and uses 
  the story's meta-data to display the stories in a Java applet.
 </description>
 <spec>
  <trigger from="WB.Stories" type="Story" />
  <trigger from="WB.Control" type="All.*" />
  <trigger from="WB.Analyses" type="Analysis.*" />
 </spec>
</module>
\end{verbatim}

It defines that it is an external module (by the \texttt{executable} tag which
can also contain more information on how to launch the module) and that it
requests triggers from the whiteboards WB.Stories, WB.Control and WB.Analyses,
but only if the type of the message is Story, All.* or Analysis.* respectively.
* matches everything, so both All.Start and All.Stop will trigger the applet.

So now that the module is defined, we can start Psyclone and it will expect the
module to be present. If it is, the messages are sent. If it isn't, the module
is temporarily deactivated until it signs on and no messages are sent to the
module.

