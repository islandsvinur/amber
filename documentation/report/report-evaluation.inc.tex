\chapter{\label{cpt:evaluation}Evaluation}

\section{Learning points}

The approach used in the development of this project is called
Constructionists Design Methodology, which is a form of incremental
design. For this project, this was a good choice, because many things were
undecided upon in the beginning and only became clear after some first
experiments.

One example would be the visualization of ``replies to a story'' which was
initially thought of as adding a particle on a tail tied to the original
story. The longer the tail, the more important a story would appear.

However, there is no connection between a reply and a story other than a
humanly constructed one (there is no field in RSS which specifies it). It
was thus impossible to create this link without going far beyond the scope
of this project.

Instead, it was chosen to give all replies their own orbiting particles
and if a story is popular, many particles appear and in this way visualize
activity in a topic.

\section{Points for improvement}

It is very hard to create a realistic planning and to estimate one's
capabilities right. Some details had to be simplified on the way to make
the project fit within the available time, while other aspects which
seemed complex were in fact solved very quickly.

